This thesis consists of three essays that attempt to provide original empirical analyses of significant problems in finance. The first essay deals with the returns of actively managed mutual funds; the second and third essays attempt to bring further understanding to ultra-high-frequency market microstructure data. The thread that binds these chapters together---if any---is the use of latent structure models. The first chapter attempts to identify unobserved populations of managers, the second looks for hidden structure in order book shapes, and the third searches for connections between order book events.

Evaluating the performance of actively managed equity mutual funds is among the most central topics in the field of finance. In the first chapter, I present a new assessment of the stock picking ability of actively managed funds that accounts for the occurrence of false positives, an issue that complicates traditional assessments. I find that while the data is consistent with a small group of alpha-generating funds, the composition of this population experiences significant annual turnover and is, therefore, difficult to identify in advance. Between 1975 and 2015, the returns to a fund selection strategy based on the classification method fail to generate alpha.

The second chapter begins a study of high-frequency limit order book data. With a view towards exploring the information content of limit orders, as opposed to market orders, I propose a factor model of order book shape. I start by building a unique dataset of Nasdaq limit order books that tracks order activity at ultra high-frequency. Analyzing over 20,000 stock-days, I find that the limit order book comprises three common factors, which I characterize as level, slope, and curvature. By combining these factors alongside price increments in a vector autoregression, I demonstrate that the factors not only explain limit order book shape but also predict returns over one-minute time intervals. In agreement with the claim that high-frequency traders serve a role in increasing market efficiency, I find a negative correlation between predictability and high-frequency trade activity.

In the third chapter, I explore a continuous-time, event-driven model of limit order book dynamics. It is the first analysis of its kind to examine the microstructure of a broad cross-section of markets, as well as the first to introduce a Bayesian framework for the study of mutually-exciting Poisson processes in order-driven microstructure models. The picture that emerges is that of a strongly self- and mutually-exciting process characterized by intensity ``spikes'' lasting mere fractions of a second. The largest of these spikes are expected to generate between 0.5 and 2.0 order book events--up to 120 times the number of expected events per second. In the typical order book, market orders demonstrate a significant influence on limit orders and cancellations, but the relationship is non-reciprocating: while limit orders and cancellations exhibit strong interactions with each other, they have no effect on the arrival of market orders. Over 99.5\% of the markets examined are stable, and in every market examined, the network model significantly improves in-sample fit relative to a baseline Poisson model. I argue that this improvement is due almost entirely to the most active 20-25\% of connections.

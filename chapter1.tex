% chapter1.tex
\chapter{Introduction}
\label{chapter:one}

My first essay considers that challenge of evaluating actively managed mutual fund managers. A key empirical issue is that of false discoveries: when evaluating thousands of funds, we are likely to find many funds that appear to deliever alpha even if the truth is that none actually do. Put simply, conventional $p$-values fail in this setting. I borrow from \citet{Barras2010} and suggest a method for classifying fund managers by imposing additional \textit{structure} on the distribution of manager skill. By making assumptions about the distribution of manager skill, I am able to calculate probabilities concerning the skill level of managers. The results reveal that if the assumed structure is correct, then a large number of over-performing funds exists measured over long and short time horizons. However, consistent with earlier studies, trying to identify which funds will perform well in the future is not profitable.

In my second essay, I analyze Nasdaq limit order book. The issue I address is the information contained in the shape of limit order books. Empirical and theoretical studies agree that limit orders are are used by informed investors, and as a result, the shape of the order book should be related to future price movements. I show that order books have low-dimensional underlying structure, and that this structure predicts future price movements.

My third essay also analyzes Nasdaq limit order book data. In this case, I consider an event driven model of order book dynamics, as opposed to the state driven model in the preceeding chapter. The model directly estimates the connection betweeen different types of order book events (limit orders, market orders, and cancellations) and distinguishes between endogenous events from exogenous events. The results highlight the importance of order book characteristics in explaining patterns in order arrivals.

An unintentional theme ties these essays together: each of the essays features a statistical model that depends on unobserved structure in some manner. In the first essay, I assume that the fund manager skill has a mixture of normals distribution; the second essay assumes that order books have hidden low dimensional linear structure; the final essay considers latent relationships between order book events. In each case, I use statistics to infer the latent structure, and then analyze the consequences for other aspects of the data.

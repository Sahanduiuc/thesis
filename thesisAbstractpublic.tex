This thesis consists of three essays that attempt to provide novel empirical analyses of important problems in finance. The first essay deals with the returns of actively managed mutual funds; the second and third essays attempt to bring further understanding to ultra high-frequency market microstructure data.

Evaluating the performance of actively managed mutual funds is a challenging task. One source of difficultly arises from the sheer number of funds: with thousands of funds, some funds are likely to appear to generate alpha even if none truly do. The first essay in this thesis investigates a novel method of accounting for such ``false discoveries'' in the search for superior returns.

The second and third essays analyze trade data from the Nasdaq stock exchange. In the second essay, I demonstrate the limit order books have characteristic ``shapes'' that predict future price movements. In the the third essay, I analyze a new event-driven model of trade data. 

The thread that binds these chapters together---if any---is the use of latent structure models. The first chapter attempts to identify unobserved populations of managers, the second looks for hidden structure in order book shapes, and the third searches for connections between order book events.

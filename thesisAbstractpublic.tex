This thesis consists of three essays that present original empirical analyses of critical problems in finance. The first essay deals with the returns of actively managed mutual funds; the second and third essays attempt to bring further understanding to the nature of ultra-high-frequency trade data. The thread that binds these chapters together is the use of latent structure models in which the researcher proposes a plausible data model, and then designs statistical methods to learn the details of this structure from data.

Evaluating the performance of actively managed mutual funds is a challenging task. One source of difficulty arises from the sheer number of funds: with thousands of funds, a small group of funds are likely to \textit{appear} to provide value purely by chance. The first essay in this thesis investigates a novel method of accounting for such ``false discoveries'' in the search for superior returns.

The second and third essays analyze data from public trading on the Nasdaq stock exchange. In the second essay, I demonstrate that fully electronic limit order books possess characteristic ``shapes'' that predict future price movements. The results of the study reinforce a crucial feature of modern exchanges, which is that limit orders reveal information about traders information. The third and final essay analyzes a new statistical model that attempts to capture essential features of order book activity. I demonstrate that this activity is characterized by a small number intense interactions taking place over brief windows of time.
